\documentclass[10pt]{beamer} % aspectratio=169 <- 

\usetheme[progressbar=frametitle, numbering=fraction]{metropolis}
\usepackage{appendixnumberbeamer}

\usepackage{booktabs}
\usepackage[scale=2]{ccicons}
\usepackage{tikz}
\usepackage{multirow}
\usepackage{pgfgantt}
\usepackage{smartdiagram}
% Checkmark
\def\checkmark{\tikz\fill[scale=0.4](0,.35) -- (.25,0) -- (1,.7) -- (.25,.15) -- cycle;} 

%Graphics path
\graphicspath{{../figures/}}

%including tikz packages
\usepackage{tikz}
\usetikzlibrary{positioning}
\tikzset{>=stealth}
\usepackage{verbatim}
\usetikzlibrary{arrows,shapes,backgrounds}

\usepackage{pgfplots}
\usepgfplotslibrary{dateplot}

\usepackage{xspace}
\newcommand{\themename}{\textbf{\textsc{metropolis}}\xspace}
\definecolor{IUcolor}{HTML}{990009}
\definecolor{LetterColor}{HTML}{495377}
\definecolor{LetterColorL}{HTML}{05185f}
\definecolor{IUlight}{HTML}{FEC5C8}

\makeatletter
% change width of progressbar
\setlength{\metropolis@titleseparator@linewidth}{2pt}
\setlength{\metropolis@progressonsectionpage@linewidth}{2pt}
\setlength{\metropolis@progressinheadfoot@linewidth}{3pt} % <-- insert desired value here

\setbeamercolor{alerted text}{fg=IUcolor}
\setbeamercolor{progress bar}{fg=LetterColorL, bg=LetterColor}
\setbeamercolor{frametitle}{bg=IUcolor}
\title{Chest Tube Securing Device}
\subtitle{Flexural Behaviour top Part CTSD}
\date{\today}
\date{}
\author{Andres Tovar Ph.D, Edwin N. Prieto Ph.D(c). Shantanu Shinde M.Sc(c)}
\institute{\textbf{Indiana University-Purdue University Indianapolis}}
\titlegraphic{\hfill\includegraphics[height=1.5cm]{IUPUILogo.pdf}}



\begin{document}



\maketitle
% --- OPTION 2 ---
\usebackgroundtemplate{
\begin{picture}(300,273)
\includegraphics[width=\paperwidth]{backgroundIUPUI.png}
\end{picture}
}%


\begin{frame}{Table of contents}
  \setbeamertemplate{section in toc}[sections numbered]
  \tableofcontents[]
\end{frame}

\section{Objectives}

\begin{frame}[fragile]{}

\end{frame}

\section{3D printing process}

\begin{frame}[fragile]{Design of experiments (Part I)}
\begin{table}
\begin{tabular}{|c|c|c|c|c|c|c|c|}
\cline{3-8} 
\multicolumn{1}{c}{} &  & \multicolumn{6}{c|}{Top part}\tabularnewline
\cline{3-8} 
\multicolumn{1}{c}{} &  & ABS & PC/ABS & Nylon & PC & PLA & SLA\tabularnewline
\hline 
\multirow{6}{*}{\rotatebox[origin=c]{90}{Bottom part}} & ABS & \checkmark & \checkmark & \checkmark & \checkmark & \checkmark & \checkmark \tabularnewline
\cline{2-8} 
 & PC/ABS & \checkmark & \checkmark & \checkmark & \checkmark & \checkmark & \checkmark \tabularnewline
\cline{2-8} 
 & Nylon & \checkmark & \checkmark & \checkmark & \checkmark & \checkmark & \checkmark \tabularnewline
\cline{2-8} 
 & PC & \checkmark & \checkmark & \checkmark & \checkmark & \checkmark & \checkmark \tabularnewline
\cline{2-8} 
 & PLA & \checkmark & \checkmark & \checkmark & \checkmark & \checkmark & \checkmark \tabularnewline
\cline{2-8} 
 & SLA & \checkmark & \checkmark & \checkmark & \checkmark & \checkmark & \checkmark \tabularnewline
\hline 
\end{tabular}

\caption{Design of Experiments for 3D printing.}

\end{table}

\end{frame}

\begin{frame}[fragile]{Design of experiments (Part I)}
% Introduce a new counter for counting the nodes needed for circling
\newcounter{nodecount}
% Command for making a new node and naming it according to the nodecount counter
\newcommand\tabnode[1]{\addtocounter{nodecount}{1} \tikz \node (\arabic{nodecount}) {#1};}

\tikzstyle{every picture}+=[remember picture,baseline]
\tikzstyle{every node}+=[inner sep=0pt,anchor=base,
minimum width=0.9cm,align=center,text depth=.25ex,outer sep=1.5pt]
\tikzstyle{every path}+=[thick, rounded corners]
\begin{table}
\begin{tabular}{|c|c|c|c|c|c|c|c|}
\cline{3-8} 
\multicolumn{1}{c}{} &  & \multicolumn{6}{c|}{Top part}\tabularnewline
\cline{3-8} 
\multicolumn{1}{c}{} &  & ABS & PC/ABS & Nylon & PC & PLA & SLA\tabularnewline
\hline 
\multirow{6}{*}{\rotatebox[origin=c]{90}{Bottom part}} & ABS & \checkmark & \checkmark & \checkmark & \checkmark & \checkmark & \tabnode{\checkmark} \tabularnewline
\cline{2-8} 
 & PC/ABS & \tabnode{\checkmark} & \checkmark & \checkmark & \checkmark & \checkmark & \checkmark \tabularnewline
\cline{2-8} 
 & Nylon & \checkmark & \tabnode{\checkmark} & \checkmark & \checkmark & \checkmark & \checkmark \tabularnewline
\cline{2-8} 
 & PC & \checkmark & \checkmark & \tabnode{\checkmark} & \checkmark & \checkmark & \checkmark \tabularnewline
\cline{2-8} 
 & PLA & \checkmark & \checkmark & \checkmark & \tabnode{\checkmark} & \checkmark & \checkmark \tabularnewline
\cline{2-8} 
 & SLA & \tabnode{\checkmark} & \checkmark & \checkmark & \checkmark & \checkmark & \tabnode{\checkmark} \tabularnewline
\hline 
\end{tabular}
\caption{Full factorial DOE for 3D printing.}
\begin{tikzpicture}[overlay]
% Define the circle paths
\draw [orange] (2.north west) -- (2.north east) -- (2.south east) -- (3.north west) -- (3.north east) -- (3.south east) -- (4.north west) -- (4.north east) -- (4.south east) -- (5.north west) -- (5.north east) -- (5.south east) -- (7.north west) -- (7.south west) -- (6.south west) -- cycle;
\draw [red] (1.north west) -- (1.north east) -- (7.north east) -- (7.north west) -- cycle;  
\end{tikzpicture}

\end{table}

\end{frame}

\begin{frame}[fragile]{Design of experiments (Part I)}
\begin{table}
\begin{tabular}{|c|c|c|c|c|c|c|c|}
\cline{3-8} 
\multicolumn{1}{c}{} &  & \multicolumn{6}{c|}{Top part}\tabularnewline
\cline{3-8} 
\multicolumn{1}{c}{} &  & ABS & PC/ABS & Nylon & PC & PLA & SLA\tabularnewline
\hline 
\multirow{6}{*}{\rotatebox[origin=c]{90}{Bottom part}} & ABS & \checkmark & \checkmark & \checkmark & \checkmark & \checkmark & \tabularnewline
\cline{2-8} 
 & PC/ABS &  & \checkmark & \checkmark & \checkmark & \checkmark & \tabularnewline
\cline{2-8} 
 & Nylon & & & \checkmark & \checkmark & \checkmark & \tabularnewline
\cline{2-8} 
 & PC & & & & \checkmark & \checkmark &  \tabularnewline
\cline{2-8} 
 & PLA & & & & & \checkmark &  \tabularnewline
\cline{2-8} 
 & SLA & & & & & & \checkmark \tabularnewline
\hline 
\end{tabular}

\caption{Total number of combinations.}

\end{table}

\end{frame}

\begin{frame}[fragile]{Flexible part}

\begin{columns}[T,onlytextwidth]
\column{0.5\textwidth}
\begin{table}
\begin{tabular}{|l|l|}
\hline
Flexible materials & Printing Method \\ \hline
TPE                & FDM             \\ \hline
TPU                & FDM             \\ \hline
Flexible           & SLA             \\ \hline
\end{tabular}
\end{table}
\column{0.5\textwidth}
\begin{figure}
\includegraphics[scale=0.25]{receptacle}
\caption{Receptacle CAD}
\end{figure}
\end{columns}
\end{frame}

\begin{frame}[fragile]{Total number of experiments}
\vspace{1cm}
\begin{columns}[T,onlytextwidth]
\column{0.5\textwidth}
\begin{table}
\begin{centering}
\begin{tabular}{|c|c|}
\hline 
ID & No.\tabularnewline
\hline 
\hline 
Top - Base & 16\tabularnewline
\hline 
Flexible & 3\tabularnewline
\hline 
Total & 48\tabularnewline
\hline 
\end{tabular}
\par\end{centering}
\caption{Total Number of experiments}
\end{table}
\column{0.5\textwidth}
\begin{figure}
\includegraphics[scale=0.025]{chesttube}
\caption{Render of the CTSD}
\end{figure}
\end{columns}
\end{frame}

\begin{frame}[fragile]{What do we want to figure out?}	
\begin{columns}[T,onlytextwidth]
\column{0.5\textwidth}
\begin{block}{Manufacturability}
\begin{itemize}
\item Printing quality.
\item Time.
\item Printing complexity.
\end{itemize}
\end{block}
\begin{block}{Mechanical properties}
\begin{itemize}
\item Bonding force.
\item Sliding force.
\item Snapping.
\end{itemize}
\end{block}
\column{0.5\textwidth}
\begin{figure}
\includegraphics[scale=0.04]{chesttube_top}
\includegraphics[scale=0.04]{chesttube_base}
\caption{Dual material parts render.}
\end{figure}
\end{columns}
\end{frame}

\section{Plastic Injection Molding}
\begin{frame}[fragile]{Design mold process}
\smartdiagramset{border color=none,
uniform color list=IUlight for 4 items,
arrow style=[-stealth’,
module x sep=5,
back arrow distance=0.75,
priority arrow height advance=1.0cm
}
\smartdiagram[priority descriptive diagram]{
  Choose the material according to the 3D print experiment, 
  Make molds in 3D printing,
  Develop the dual injecting process,  
  Develop the final mold 
  }
\end{frame}

\begin{frame}[fragile]{Methodology}
\begin{center}
\smartdiagramset{planet color=IUcolor,
distance planet-satellite=2.8cm
}
\smartdiagram[circular diagram:clockwise]
{Simulate Physical Parameters, Build a 3D print mold, Set up the process, Run, Analyze, Modify~/\\ Add,Check}
\end{center}
\end{frame}
\section{Numerical Analysis process}
\begin{frame}[fragile]{Finite Element Analysis}
\begin{columns}[T,onlytextwidth]
	\column{0.5\textwidth}
	\begin{block}{Purpose I}
	\begin{itemize}
		\item {To determine max stresses.}
		\item {To determine the product life (fatigue).}
		\item {To make structural optimization.}
	\end{itemize}
	\end{block}
	\begin{block}{Purpose II}
	\begin{itemize}
		\item {CTSD experiments.}
		\item {To avoid mold trials.}
	\end{itemize}
	\end{block}
	\column{0.5\textwidth}
	\begin{figure}
	\vspace{1cm}
	\includegraphics[scale=0.25]{CTSDsimulation}
	\caption{Explicit Simulation CTSD}
	\end{figure}
\end{columns}
\end{frame}

\section{Testing process}

\begin{frame}[fragile]{Sliding force}
\begin{columns}[T,onlytextwidth]
\column{0.5\textwidth}
\metroset{block=fill}
	\begin{figure}
	\vspace{1cm}
	\includegraphics[scale=0.03]{test-CTHD}
	\caption{Sliding force testing.}
	\end{figure}
\column{0.5\textwidth}
	\begin{figure}
	\vspace{1cm}
	\includegraphics[scale=0.5]{Test1ChestHold}
	\caption{Force Displacement curvature}
	\end{figure}
\end{columns}
\end{frame}

{%

\section{Schedule}
\begin{frame}[fragile]{Schedule}

\begin{ganttchart}[
time slot format=isodate-yearmonth,
x unit=1.0cm,
y unit title=0.7cm,
y unit chart=0.8cm,
vgrid,
time slot unit=month,
%compress calendar,
title/.append style={draw=none, fill=IUcolor},
title label font=\footnotesize\sffamily\bfseries\color{white},
title label node/.append style={below=-1.6ex},
title left shift=.05,
title right shift=-.05,
title height=1,
bar/.append style={draw=none, fill=green!75},
bar height=.6,
bar label font=\normalsize\color{black!50},
group right shift=0,
group top shift=.6,
group height=.3,
group peaks height=.2,
bar incomplete/.append style={fill=IUlight},
]{2018-07}{2018-12}
%\gantttitlecalendar{year}{months} \\
\gantttitle[]{2018}{5} \\                 % title 
	\gantttitle{Aug}{1}
    \gantttitle{Sep}{1}
    \gantttitle{Oct}{1}
    \gantttitle{Nov}{1}
    \gantttitle{Dic}{1}\\
\ganttset{progress label text={}, link/.style={black, -to}}
\ganttgroup{Pre-manufacturing Line}{2018-07}{2018-12}\\ 
\ganttbar[progress=10, name=T1A]{3D printing experiment.}{2018-07}{2018-10} \\
\ganttbar[progress=0, name=T1A]{Testing 3D printing}{2018-09}{2018-10} \\
\ganttbar[progress=0, name=T1A]{Injection Molding}{2018-09}{2018-11} \\
\ganttbar[progress=0, name=T1A]{Testing IM prototypes}{2018-10}{2018-12} \\
\ganttbar[progress=0, name=T1A]{Preparing one journal paper.}{2018-08}{2018-12} \\
\ganttset{link/.style={black}}
%\ganttlink[link mid=.4]{pp}{T1A}
%\ganttlink[link mid=.159]{pp}{T2A}
\end{ganttchart}
\end{frame}
{\setbeamercolor{palette primary}{fg=black, bg=yellow}
\begin{frame}[standout]
  Questions?
\end{frame}
}

%\appendix


%\begin{frame}[allowframebreaks]{References}

%  \bibliography{demo}
%  \bibliographystyle{abbrv}

%\end{frame}

\end{document}
